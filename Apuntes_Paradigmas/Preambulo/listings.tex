%%%%%%%%%%%%%%%%%%%%%%
% Estilo del entorno listings para insertar código fuente
\lstdefinestyle{mystyle}{
	backgroundcolor = \color{gray!20},	            % Color de fondo
	commentstyle = \color{myblue},					% Color de los comentarios
	keywordstyle = \color{mypurple},				% Color de las palabras reservadas
	stringstyle = \color{myred}\itshape,			% Color de las cadenas de caracteres
	basicstyle =\ttfamily,			                % Formato del texto
	breakatwhitespace = false,				
	breaklines = true								% Cambiar de línea
	keepspaces = true,							    % Mantener los espacios del editor en el pdf
	numbers = left,								    % Posición de los números de línea
	numbersep = 5pt,								% Separación de los números de línea
	numberstyle = \color{gray},					    % Color de los números de línea
	showspaces = false,							        % Mostrar los espacios
	showstringspaces = false,						% Mostrar los espacios en las cadenas de caracteres
	showtabs = false,								% Mostrar las tabulaciones
	tabsize = 3,									% Tamaño de las tabulaciones
	frame = lt,                                     % Tipo de margen
	captionpos = t							    	% Posición del título del listing
}
\renewcommand{\lstlistingname}{Código}		% "listing" ---> Nuevo nombre
\renewcommand{\lstlistlistingname}{Lista de prácticas}	% "List of listings" ---> Nuevo nombre
%%%%%%%%%%%%%%%%%%%%%%
